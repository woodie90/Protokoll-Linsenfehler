\section{Einleitung}

Die Ausbreitung eines Lichtstrahls lässt sich prinzipiell durch folgende Geradengleichung beschreiben:
\begin{align*}
y(z)=y_1 + z \cdot \tan \Theta
\end{align*}

Für kleine Winkel gilt $\tan\Theta \approx \Theta$ bzw. $\sin \Theta \approx \Theta $.Dies lässt sich leicht durch die Reihenentwicklung der Winkelfunktionen zeigen:
\begin{align*}
\sin \Theta = \Theta - \frac{1}{3!}\Theta^3 + \frac{1}{5!}\Theta^5 - ... \\
\tan \Theta = \Theta + \frac{1}{3}\Theta^3 + \frac{2}{15}\Theta^5 + ...
\end{align*}

Die Geradengleichung kann folglich für viele optische Systeme, deren Strahlen sich nahe der optischen Achse ausbreiten, durch eine paraxiale Näherung 
\begin{align*}
y(z)=y_1 + z \cdot \Theta
\end{align*}
ausgedrückt werden.(Theorie 1. Ordnung) 

Berücksichtigt man zusätzlich noch den 2. Summanden der Reihenentwicklung ( kubische Abhängigkeit vom Winkel) ergibt sich die Theorie der 3. Ordnung :
\begin{align*}
y(z)=y_1 + z \cdot (\Theta + \frac{1}{3}\Theta^3)
\end{align*}
Im Vergleich zur paraxialen Näherung können hierbei 6 einfache Aberrationen unterschieden werden, 5 davon sind achromatisch:
\begin{description}
	\item[sphärische Aberration]
	\item[Koma]
	\item[Astigmatismus]
	\item[Bildfeldwölbung]
	\item[Verzeichnung]
	\item[chromatische Aberration]
\end{description}