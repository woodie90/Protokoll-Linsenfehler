\documentclass[11pt,fleqn]{article}
\usepackage{pgf,tikz}
\usepackage[ngerman]{babel}
\usepackage[utf8]{inputenc}
\usepackage[T1]{fontenc}
\usepackage{float}
\usepackage{mathtools}
\usepackage{fancyhdr}
\usepackage[margin=1.2in]{geometry}
\usepackage{graphicx}
\usepackage{lmodern}
\usepackage{circuitikz}
\usepackage{pgfplots}
\usepackage{pgfplotstable}
\usepackage{amsmath}
\usepackage{amssymb}
\usepackage{amsfonts}
\usepackage{siunitx}
\usepackage[backend=biber, citestyle=alphabetic, style=alphabetic]{biblatex}
\usepackage{pdfpages}

\bibliography{literatur.bib}

\parindent 0pt
\parskip 10pt
\newcommand{\degre}{\ensuremath{^\circ}}

\title{Technische Optik \\ Praktikum Linsenfehler}
\date{4. Juni 2015}
\author{Hans Herrmann \and Felix Kayser \and Hermann Pommerenke \and Tino Steinmetz}

\begin{document}
	\pagenumbering{gobble}
	\begin{figure}[t]
	    \centering
	    \includegraphics[width=115mm]{img/UNI-Logo_Siegel_4c_115mm_07.png}
	\end{figure}

	\maketitle
	
	\thispagestyle{empty}

	\newpage
	\pagenumbering{Roman}
	\pagestyle{headings}
	\tableofcontents
	
	\everymath{\displaystyle} % Jede Formel im Displaystyle
	
	\newpage
	\pagenumbering{arabic}

	\section{Einleitung}

Die Ausbreitung eines Lichtstrahls lässt sich prinzipiell durch folgende Geradengleichung beschreiben:
\begin{align*}
y(z)=y_1 + z \cdot \tan \Theta
\end{align*}

Für kleine Winkel gilt $\tan\Theta \approx \Theta$ bzw. $\sin \Theta \approx \Theta $.Dies lässt sich leicht durch die Reihenentwicklung der Winkelfunktionen zeigen:
\begin{align*}
\sin \Theta = \Theta - \frac{1}{3!}\Theta^3 + \frac{1}{5!}\Theta^5 - ... \\
\tan \Theta = \Theta + \frac{1}{3}\Theta^3 + \frac{2}{15}\Theta^5 + ...
\end{align*}

Die Geradengleichung kann folglich für viele optische Systeme, deren Strahlen sich nahe der optischen Achse ausbreiten, durch eine paraxiale Näherung 
\begin{align*}
y(z)=y_1 + z \cdot \Theta
\end{align*}
ausgedrückt werden.(Theorie 1. Ordnung) 

Berücksichtigt man zusätzlich noch den 2. Summanden der Reihenentwicklung ( kubische Abhängigkeit vom Winkel) ergibt sich die Theorie der 3. Ordnung :
\begin{align*}
y(z)=y_1 + z \cdot (\Theta + \frac{1}{3}\Theta^3)
\end{align*}
Im Vergleich zur paraxialen Näherung können hierbei 6 einfache Aberrationen unterschieden werden, 5 davon sind achromatisch:
\begin{description}
	\item[Sphärische Aberration] bezeichnet den Abstand zwischen dem axialen Schnittpunkt eines gebrochenen Strahls und dem paraxialen Brennpunkt $F'$ eines optischen Systems. Durch diesen Abstand, innerhalb dessen die gebrochenen Strahlen die optische Achse schneiden, lässt sich mit der betreffenden Optik keine Absolut scharfe Abbildung erzeugen. \cite[416ff.]{hecht2014optik}
	\item[Koma] bezeichnet einen Abbildungsfehler der bei Objektpunkten, die sich nicht auf der optischen Achse befinden, zu einem zum Außenrand der Optik gerichteten "`Schweif"' führt. Sie resultiert daraus, dass die Hauptebenen eines optischen Systems nur in achsennaher Umgebung ebene Flächen sind. Dadurch unterscheiden sich die Äquivalentbrennweiten und Transversalvergrößerungen für Strahlen, die die außeraxiale Bereiche einer Linse durchlaufen. \cite[423ff.]{hecht2014optik}
	\item[Astigmatismus] tritt bei Objektpunkten auf, die nicht auf der optischen Achse liegen. Diese Erscheinung basiert darauf, dass sich die Geometrie der einfallenden Strahlen in zwei senkrecht aufeinander stehende Ebenen aufteilen lässt, die unterschiedliche Brennweiten besitzen. Die Meridionalebene ist als die Ebene definiert, die sowohl den Hauptstrahl als auch die optische Achse enthält. Die Sagittalebene ist demzufolge die Ebene, die den Hauptstrahl enthält und senkrecht auf der Meridionalebene steht. Ein mit dem Astigmatismus behaftetes optisches System besitzt keine einzelnen Brennpunkt, sondern zwei senkrecht zueinander stehende Brennlinien. \cite[428ff.]{hecht2014optik}
	\item[Bildfeldwölbung]
	\item[Verzeichnung]
	\item[chromatische Aberration]
\end{description}
	
	\clearpage
	\section{Versuchsaufbau}

\begin{figure}[h!]
	\includegraphics[width=\linewidth]{img/versuchsaufbau.png}
	\caption{Schematischer Versuchsaufbau}
	\label{fig:versuchsaufbau}
\end{figure}

Der Versuchsaufbau wurde mithilfe des Mikro-Bank-Systems Linos realisiert. Der lineare Aufbau bestand aus einer Lichtquelle (Weißlicht), einer Streuscheibe, einem Farbfilter, einem Objekt und einer zu untersuchenden Linse. 

Das Licht der Quelle wurde zuerst in der Streuscheibe gebrochen, um diffuses Licht zu erhalten. Der darauf folgende Farbfilter war einstellbar: Es war sowohl möglich das Licht ungefiltert passieren zu lassen, als auch einen roten, blauen oder grünen Filter anzuwenden. Direkt hinter dem Filter war das Beobachtungsobjekt angebracht. Hierfür eigneten sich beispielsweise ein Siemensstern, eine Skala oder ein grobes Gitter. Hinter dem Objekt wurden Linsen auf das Mikro-Bank-System gesteckt. Diese Linsen zeigten jeweils einen der zu untersuchenden Abbildungsfehler besonders deutlich. Um diesen Fehler sichtbar zu machen, wurde der Fokus über die Position der Linsen eingestellt. 

Die Abbildungen wurden auf einen weißen Papierschirm projiziert, was gegenüber der Verwendung eines CCD-Sensors den Vorteil bot, dass die Abbildungsfehler mit den Augen beobachtet werden konnten. Somit wurde das Auswählen der für den gewünschten Fehler geeignetsten Linse und das Fokussieren vereinfacht.

Nachdem eine für das Protokoll geeignete Abbildung auf dem Schirm realisiert werden konnte, wurde dieser mit einer Spiegelreflexkamera fotografiert. Hierbei war ein manueller Fokus und Weißabgleich zu benutzen.

\begin{figure}[h!]
	\includegraphics[width=\linewidth]{img/Versuchsaufbau.jpg}
	\caption{Der Versuchsaufbau}
	\label{fig:versuchsaufbaureal}
\end{figure}
	
	\clearpage
	\section{Auswertung}

\subsection{Sphärische Aberration}

\subsection{Koma}

\begin{figure}[htb]
	\begin{minipage}[t]{0.32\textwidth}
		\includegraphics[width=\linewidth]{img/Koma/Prakt_Linsenfehler_2015_06_04_097}
		\label{fig:koma_stark}
		\caption{Starkes Koma am Außenrand der Linse}
	\end{minipage}
	\hfill
	\begin{minipage}[t]{0.32\textwidth}
		\includegraphics[width=\linewidth]{img/Koma/Prakt_Linsenfehler_2015_06_04_096}
		\label{fig:koma_schwach}
		\caption{Schwaches Koma nahe der optischen Achse}
	\end{minipage}
	\hfill
	\begin{minipage}[t]{0.32\textwidth}
		\includegraphics[width=\linewidth]{img/Koma/Prakt_Linsenfehler_2015_06_04_099}
		\label{fig:koma_korrigiert}
		\caption{Abbildung mit Korrektur der Koma}
	\end{minipage}	
\end{figure}

\subsection{Astigmatismus}

\begin{figure}[htb]
	\begin{minipage}[t]{0.48\textwidth}
		\includegraphics[width=\linewidth]{img/Astigmatismus/Prakt_Linsenfehler_2015_06_04_087_saggital}
		\label{fig:astigmatismus_saggital}
		\caption{Abbildung der Saggitalebene}
	\end{minipage}
	\hfill
	\begin{minipage}[t]{0.48\textwidth}
		\includegraphics[width=\linewidth]{img/Astigmatismus/Prakt_Linsenfehler_2015_06_04_088_meridional}
		\label{fig:astigmatismus_meridional}
		\caption{Abbildung der Meridionalebene}
	\end{minipage}
	
	\vspace{0.5cm}
	
	\begin{minipage}[t]{0.48\textwidth}
		\includegraphics[width=\linewidth]{img/Astigmatismus/Prakt_Linsenfehler_2015_06_04_089_mittenfokus}
		\label{fig:astigmatismus_mittenfokus}
		\caption{Fokus zwischen meridionaler und saggitaler Abbildung}
	\end{minipage}
	\hfill
	\begin{minipage}[t]{0.48\textwidth}
		\includegraphics[width=\linewidth]{img/Astigmatismus/Prakt_Linsenfehler_2015_06_04_086}
		\label{fig:astigmatismus_korrigiert}
		\caption{Korrektur des Astigmatismus}
	\end{minipage}
\end{figure}

\subsection{Bildfeldwölbung}

\begin{figure}[htb]
	\begin{minipage}[t]{0.48\textwidth}
		\includegraphics[width=\linewidth]{img/Bildwoelbung/Prakt_Linsenfehler_2015_06_04_074}
		\label{fig:bildwoelbung_aussen}
		\caption{Unschärfe am Außenrand des Gitters}
	\end{minipage}
	\hfill
	\begin{minipage}[t]{0.48\textwidth}
		\includegraphics[width=\linewidth]{img/Bildwoelbung/Prakt_Linsenfehler_2015_06_04_075}
		\label{fig:bildwoelbung_korrigiert}
		\caption{Unschärfe in der Mitte des Gitters}
	\end{minipage}
\end{figure}

\begin{figure}[htb]
	\includegraphics[width=\linewidth]{img/Bildwoelbung/Prakt_Linsenfehler_2015_06_04_076}
	\label{fig:bildwoelbung_korrigiert}
	\caption{Korrektur der Bildfeldwölbung durch gekrümmten Projektionsschirm}
\end{figure}

\subsection{Verzeichnung}

\begin{figure}[htb]
	\begin{minipage}[t]{0.48\textwidth}
		\includegraphics[width=\linewidth]{img/Verzeichnung/Prakt_Linsenfehler_2015_06_04_082}
		\label{fig:verzeichnung}
		\caption{Am Rand des Gitters erkennbare Krümmung}
	\end{minipage}
	\hfill
	\begin{minipage}[t]{0.48\textwidth}
		\includegraphics[width=\linewidth]{img/Verzeichnung/Prakt_Linsenfehler_2015_06_04_083}
		\label{fig:verzeichnung_korrigiert}
		\caption{Korrektur der Verzeichnung}
	\end{minipage}
\end{figure}

\subsection{Chromatische Aberration}

\begin{figure}[htb]
	\begin{minipage}[t]{0.32\textwidth}
		\includegraphics[clip=true, trim=700px 950px 900px 250px, width=\linewidth]{img/ChromAbb/Prakt_Linsenfehler_2015_06_04_068}
		\label{fig:cm_blau}
		\caption{Fokussierte Abbildung der blauen Wellenlängen (gewöhnliche Linse)}
	\end{minipage}
	\hfill
	\begin{minipage}[t]{0.32\textwidth}
		\includegraphics[clip=true, trim=707px 950px 907px 250px, width=\linewidth]{img/ChromAbb/Prakt_Linsenfehler_2015_06_04_069}
		\label{fig:cm_gruen}
		\caption{Leicht defokussierte Abbildung der grünen Wellenlängen (gewöhnliche Linse)}
	\end{minipage}
	\hfill
	\begin{minipage}[t]{0.32\textwidth}
		\includegraphics[clip=true, trim=700px 950px 900px 250px, width=\linewidth]{img/ChromAbb/Prakt_Linsenfehler_2015_06_04_070}
		\label{fig:cm_rot}
		\caption{Defokussierte Abbildung der roten Wellenlängen (gewöhnliche Linse)}
	\end{minipage}
		
	\vspace{0.5cm}
	
	\begin{minipage}[t]{0.32\textwidth}
		\includegraphics[clip=true, trim=700px 950px 900px 250px, width=\linewidth]{img/ChromAbb/Prakt_Linsenfehler_2015_06_04_071}
		\label{fig:cm_blau_achromat}
		\caption{Fokussierte Abbildung der blauen Wellenlängen (Achromat)}
	\end{minipage}
	\hfill
	\begin{minipage}[t]{0.32\textwidth}
		\includegraphics[clip=true, trim=700px 950px 900px 250px, width=\linewidth]{img/ChromAbb/Prakt_Linsenfehler_2015_06_04_072}
		\label{fig:cm_gruen_achromat}
		\caption{Fokussierte Abbildung der grünen Wellenlängen (Achromat)}
	\end{minipage}
	\hfill
	\begin{minipage}[t]{0.32\textwidth}
		\includegraphics[clip=true, trim=700px 950px 900px 250px, width=\linewidth]{img/ChromAbb/Prakt_Linsenfehler_2015_06_04_073}
		\label{fig:cm_rot_achromat}
		\caption{Fokussierte Abbildung der roten Wellenlängen (Achromat)}
	\end{minipage}
\end{figure}



	
	\clearpage
	\section{Anhang}

Hans erforscht GitHub :)
	
	\clearpage
	\printbibliography
\end{document}
